\pagestyle{plain}
\lfoot{}
\cfoot{\zihao{5}\thepage}
\rfoot{}
\pagenumbering{Roman} %目录页码为罗马数字
	\begin{center}
		\heiti \zihao{3} 2022年首届“钉钉杯”大学生大数据挑战赛论文
		
		\section*{\zihao{-3}\heiti 基于XGB、LGB、随机森林的Stacking诈骗预测模型}
		\section*{\zihao{3}\heiti 摘\ \ \ \ \ \ \ \ 要}
	\end{center}


随着互联网的迅猛发展以及移动支付的普及,银行卡电信诈骗的问题愈加严峻。本文通对银行卡交易信息数据进行EDA、特征工程、建立模型、模型调参、模型融合、模型评价,建立了基于XGB、LGB、随机森林的Stacking诈骗预测模型,最终选择Stackin作为预测模型,模型准确率为0.9999987,F-1值为0.999885。最后给客户提出建议\textbf{:用户如果能够使用pin交易,那么将会使欺诈发生的概率趋近于0;当用户发现当前交易是往期交易的数倍时,用户需要谨慎思考当前是否遭遇了银行卡欺诈行为}。从Kaggle获取我们的代码:\href{github首页}{https://www.kaggle.com/code/zhouhua2022/card-transdata}

\textbf{EDA}:导入需要用到的包,读取数据;查看数据基本信息,查看7个分类属性数据的分布情况,查看标签在7个分类属性条件下的分布情况。


\textbf{数据预处理}:第一步,对数据进行\textbf{异常值处理}。查看3个连续型属性的分布,发现只有极少的值是异常值,考虑到异常数值对应属性在实际生活中是切实存在的,并且异常值处理前后平均值与方差的变化在5\%以内,因此不对数据进行异常值处理。第二步,对三个连续型属性数据进行\textbf{标准化}。第三步,对4个分类属性数据进行\textbf{独热编码}。第四步,\textbf{选择分类特征}。查看属性的相关系数矩阵,发现属性间相关性小,属性独立性好,故没有进一步对数据降维的必要。第五步,\textbf{数据集划分}。对数据集进行划分,训练集与测试集比例为7:3。

\textbf{模型建立}:使用LGB,XGB,KNN,线性SVC,朴素贝叶斯,决策树随机森林,感知机,逻辑回归,随机剃度下降\textbf{10个模型}对数据进行拟合并查看在测试集上的准确率,其中决策树,随机森林的准确率达到了\textbf{100\%}


\textbf{模型调参与模型融合}:第一步,\textbf{模型调参}。选取XGB、LGB、KNN、决策树、随机森林5个准确度最高的模型,使用网格搜索调节参数。第二步,\textbf{模型融合}。从调节参数之后的5个模型中选出三个表现最好的XGB、LGB、随机森林建立Stacking模型。



\textbf{模型评价与可视化}:计算6个模型的$precision$、$recall$、$f1-score$,画出6个模型的混淆矩阵热力图、ROC曲线,最后根据F1分数值选出最优模型为Stacking模型。


\textbf{提出建议}:通过标签与分类属性的数据分布情况、标签与分类属性的相关性系数、决策树特征重要性分析,给客户提出合理建议。



\par


\par
%空一行

\noindent \zihao{-4}{\heiti 关键词:}\textbf{XGB};\textbf{随机森林};\textbf{Stacking模型};\textbf{F1分数值}

\clearpage



%目录样式
% \renewcommand{\cftdot}{\ensuremath{\ast}}
\titlecontents{section}
[1em]
{\heiti\zihao{-4}}%
{\contentslabel{1em}}%
{}%
{\titlerule*[0.32pc]{$\cdot$}\contentspage}
\titlecontents{subsection}
[3.5em]
{\songti\zihao{-4}}%
{\contentslabel{1.7em}}%
{}%
{\titlerule*[0.32pc]{$\cdot$}\contentspage}
\titlecontents{subsubsection}
[6em]
{\songti\zihao{-4}}%
{\contentslabel{2.5em}}%
{}%
{\titlerule*[0.32pc]{$\cdot$}\contentspage}
\titlecontents{section*}
[0em]
{\heiti\zihao{-4}}%
{\contentslabel{1em}}%
{}%
{\titlerule*[0.32pc]{$\cdot$}\contentspage}
\renewcommand{\cftsecleader}{\cftdotfill{\cftdotsep}}
\newcommand\mydot[1]{\scalebox{#1}{.}}
\renewcommand\cftdot{\mydot{0.8}}
\renewcommand\cftdotsep{1}

%\captionsetup[contentsname={\zihao{3}目\ \ \ \ \ \ \ \ 录}]
\renewcommand{\contentsname}{\zihao{3}目\ \ \ \ \ \ \ \ 录}
\renewcommand{\cftsecfont}{\heiti\zihao{-4}} %设置section条目的字体
\renewcommand{\cftsecfont}{\heiti} %设置section条目的字体
%\renewcommand{\cftsecpagefont}{\normalfont}
\renewcommand{\cftsubsecfont}{\songti\zihao{-4}} %设置subsection条目的字体
\rhead{}
\tableofcontents